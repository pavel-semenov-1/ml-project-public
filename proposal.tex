% Author: Pavel Semenov
% Date: 06.01.2023

\documentclass{article}
\usepackage{array}
\usepackage{longtable}
\usepackage{float}
\title{Project proposal}
\author{Pavel Semenov}
\date{06.01.2023}
\begin{document}
\maketitle

\section*{What is the problem?}

I am currently employed at the IBM as a DevOps engineer. I have almost no project management responsibilities, but
I am sometimes asked by project manager to give an estimation of time required to complete the task. It is always
very difficult for me to give an accurate estimate. Thus, I would like to produce a model, which might be able to
approximate the time needed for the task to be finished. I am not sure there is a strong correlation between the
task parameters and the resolution time, but I am willing to discover it if it exists. We'll see.

\section*{What is my dataset?}

I have a collection of all resolved tasks from one of the IBM projects, which have non-zero time spent parameter.
I have exported it as an XML with \textbf{28791} work items. Each work item has the following parameters:

\begin{center}
\begin{longtable}[H]{ | m{5em} | m{10em}| m{20em} | }  
    \hline
    Name & Description & Example \\
    \hline\hline
    timeSpent & The time spent on the task in milliseconds & 144000000 \\
    \hline
    summary & Short summary of the work item & FST-4320 - Falscher Ablauf beim Versand XJustiz Nachricht nachricht.vag.fehler Version 3.3.x mit dem gültigem Signaturstatus, mit fehlerhafte Schematronprüfung, mit erfolgreiche Plausibiltitätsprüfung \\
    \hline
    description & Multiline description of the work item & Falscher Ablauf beim Versand XJustiz Nachricht nachricht.vag.fehler Version 3.3.x mit dem gültigem Signaturstatus, mit fehlerhafte Schematronprüfung, mit erfolgreiche Plausibiltitätsprüfung

    Es existiert ein FAM VA Verfahren mit Auskunftsersuchen. Eine XJustiz Nachricht nachricht.vag.fehler Version 3.3.1 mit Signaturstatus unknown, mit fehlerhafte Schematronprüfung und erfolgreiche Plausibilitätsprüfung wurde an Postfach gesendet.  
    Es läuft der AF002 Normalablauf (Schritte 1-9). Im Schritt 10. stellt System fest, dass Regelverstöße bei der Schematronprüfung vorliegen. 
    
    FEHLER
    IST: Aus dem Verfahren wurde die XJustiz Nachricht nachricht.vag.fehler gesendet und die eingehende XJustiz Nachricht nachricht.vag.fehler landete im Verfahren. 
    SOLL: Es läuft der AF002 Normalablauf (Schritte 1-9). Im Schritt 10. stellt System fest, dass Regelverstöße bei der Schematronprüfung vorliegen und läuft den alternativen Ablauf AF10a erfolgreich weiter.  
    Es wird eine Rückmeldung, XJustiz Nachricht nachricht.vag.fehler Version 3.3.1 mit Verarbeitungsergebnis: FE00110 Beim Validieren gegen das Schema ist ein Fehler aufgetreten! Details: "Fehlermeldung aus Parser".) erstellt und versendet. Das System führt die Aktivität "067\_Nachricht aus Zwischenspeicher loeschen" ("Nachrichtentyp") für die eingegangene Nachricht aus. Die eingehende XJustiz Nachricht wird nichts in Verfahren gespeichert, sondern wird gelöscht. 
    
     \\
    \hline
    priority & Priority assigned to the work item. Ranged from 1 to 5 & Prio 2 \\
    \hline
    severity & The severity of the work item. Either Blocker or Normal & Blocker \\
    \hline
    category & Category of the work item. Might be Release-Test, KM (KonfigurationManagement), Maintenance, etc. (fixed collection of values) & Release-Test \\
    \hline
    type & Type of the work item. Internal Defect/Issue/Action/etc. (fixed collection of values) & Issue \\
    \hline
\end{longtable}
\end{center}

\section*{What is my plan?}
I haven't given this part a lot of though yet. However, I can think of a few steps.
Firstly, I would need use some NLP technique (Word2Vec?) to transform the text into vector. Then, I can use it along
with other parameters as an input to my Neural Network. Lastly, I would play with its parameters to get the best results.
I would greatly appreciate your feedback on this proposal and plan in particular, because my ML knowledge is not sufficient
eough to say if this can be done the proposed way.

\end{document}